
\newcommand{\systemfit}{\textit{system fit}\xspace}
%%%%%%%%% PROJECT SUMMARY -- 1 page, third person
% 
% e.g:  "The PI will prove" not "I will prove"

	%Below are the pagination, font size, spacing and margin
	%instructions for NSF proposals: \\
	%
	%FastLane does not automatically paginate a proposal.
	%Each section of the proposal must be individually
	%paginated prior to upload to the system. \\
	%
	%Use Computer Modern family of fonts at a font size of 11 points or
	%larger. A font size of less than 10 points may be used for mathematical
	%formulas or equations, figure, table or diagram captions and when
	%using a Symbol font to insert Greek letters or special characters.
	%The text must still be readable. The use of small type not in compliance with the NSF guidelines
%may be grounds for NSF to return the proposal without review. \\
%
%No more than 6 lines of text within a vertical space of 1 inch. \\
%
%Margins, in all directions, must be at least an inch. \\
%
%

% I want to make storage systems easier to configure and easier to maintain
% correctly

% currently, there is a lot of institutional knowledge tied into "tuning" a
% system for optimal performance for a given set of applications.  there are 2
% problems with this: first, the tuning takes time and can be highly fragile to
% small shifts in usage.  second, the tuning requires low level optimization
% that can't be done on shared cloud storage, so often a "best fit" or "good
% enough" generic approach is used, potentially using a coarse, high-level
% characterization such as "archival" that we know is poorly defined /cite~\ian
% 
% additionally, this tuning is specific to a workload and infrastructure (both
% hardware and software) configuration.  Transferring knowledge is difficult
% even between similar scenarios, such as banks using GPFS.  As a workload
% changes significantly, there is no systematic way to transfer previous
% configuration optimality knowledge; you have to rederive it all over again.
% 
% Finally, there is no way to determine a rigorous cost-benefit analysis of
% "should i change some part of my infrastructure" as different storage media
% and filesystem techs become available.  I want to answer "what will inline
% dedup cost me?", "should i add an SSD?", "Do I need more write-cache?".  These
% are all individual large projects in their own right, but what they all have
% in common is a need for a fundamental measure of "what's going on in my
% system", which is our workload blueprint.
%
% Not a fingerprint, because those don't change.  Imagine more of a scaffold, a
% set of instructions that give an idea of what the workload will look like at
% any given time but doesn't fully specifcy it.  Maybe just skeleton?  Skeletons
% in Storage!!
%
%  -> what if there existed a quantitative taxonomy of storage workloads?
%  -> what if we could automatically match a trace to a workload type?
%  -> What if we could decouple workload features such that system "fit" could
%  be determined?

% Storage system tuning: Identifying SKELETONS!!!

%\required{Project Summary}
\subsection*{Project Summary}
%\begin{center}
%A critical tool for accelerating performance in large-scale distributed systems---challenges 
%that may cost millions of dollars and even bankrupt smaller businesses---is the cache.
%In contrast to traditional caches, the modern application-level caches face 
%The shift requires rethinking to tackle the performance challenges of tomorrow's systems.
%With systems facing mounting scalability challenges, 
%time is ripe to rethink the cache abstraction and 
%to address how these fundamental assumptions have changed. 
% ---
%Caches enable large-scale systems to respond quickly to common queries and shield back-end databases from disruptively high loads,
%but are ignorant of their system environment, contents and costs.
%This five-year career development proposal is an integrated
%research, outreach and education program that focuses on 
%rethinking the cache design and abstraction to put
%smart caches at the center of accelerating tomorrow's large-scale systems.

%Modern data management systems are typically optimized to serve a specific type
%of workload. 

%Understanding how storage systems are used is often the difference between 
Data centers account for between 3-4\% of annual global energy                                          
consumption~\cite{nrdc}.  In the United States alone, data centers are wasting                          
$3.9\times10^{10}$ kW/h, or over \$3.8 billion dollars, of power due to a                               
combination of peak provisioning, poor workload prediction, and competing
storage goals~\cite{masanet,nrdc}. 
%This number is only going to increase as                            
%more services move into the cloud while retaining more data for increasingly                            
%indefinite amounts of time~\cite{nrdc,baker2006fresh}. 

Characterization of workload                    
features will lead to better dynamic provisioning of data centers, which will                           
reduce the waste of mismatched provisioning between storage workloads and data
center design.  
%Tuning distributed storage systems as workloads shift over time allows for intelligent trade-offs between %factors such as power, cost, and data reliability, allowing datacenters to better simultaneously serve %multiple clients.
%Modern applications on shared cloud storage have wildly different requirements for consistency, cost, and %latency, among other factors.  Storage elasticity is critical for meeting the requirements of different %workloads as they share the same physical storage, and as the workloads themselves shift over time.
Currently, many storage tuning guidelines depend on labeling workloads with high level categories such as ``Archival,'' ``Streaming,'' or ``High Performance.'' %Traditional workload labels such as ``archival'' and ``HPC'' 
These labels are poorly
understood and inconsistently applied.  As usage of systems has evolved, the common
language to describe this usage has stagnated in favor of institutional knowledge and reactive storage~\cite{TK}.
%
%To better understand how
%workload type translates into system design requirements, a combination
%of longitudinal analysis and statistical feature extraction to categorize
%workload traces and study how the properties of classical workload types, such
%as the ``write-once, read-maybe'' assumption for archives, have evolved over
%time is necessary.  

This proposal outlines the groundwork for a research program to supplant these broad categories with a set of \mWs defined by learned \textit{metrics}.  Metrics, such as ``periodicity'' or ``read locality'' will be mapped to \textit{actionable hints} for administrators to improve workload-aware storage provisioning.
 
%A small, fixed set of \mws that are sufficiently distinct, measured by maximizing entropy between %workloads, will  to that
%can be combined to form unique signatures for any specific workload type.  These signatures, in turn, can %provide specific, quantitative suggestions to administrators tasked with tuning storage systems.

%These signatures will provide quantitative
%measures to automatically configure storage systems and improve metrics such as power, availability, and
%performance by mathematically relating storage algorithms with workload
%properties using a notion of \systemfit.

The program addresses three mutually reinforcing \textbf{research objectives}.
%collectively called \smartcache{}.
%comprised of three mutually reinforcing components:
\begin{myitemize}
%\item[(i)] 

\item[1.]
%\item[(ii)] 
%\textbf{(ii)}
\emph{What if there was a parameterized taxonomy of \textit{model workloads} for
storage?}
 % this includes some idea of global metrics
 % gaurantees about temporal durability of said metrics
 % 

The PI proposes to identify a set of canonical \textit{model workloads} to serve as basis vectors for the space
of workload characteristics.  These models will be developed by combining domain expertise and machine learning techniques to ascertain what
metrics are most relevant across and within different usage scenarios, such as
GPFS deployments or high performance scientific experiments.  
%Once a set of
%metrics are identified, the PI will generate a dendogram of metric combinations such
%that properties of individual workloads as well as likely evolution of
%workloads becomes easier to predict.  

%This will be experimentally validated by
%generating traces using the properties of the model workloads and showing that
%these model workloads capture enough real workload behaviour that the system
%response is predictable.  
Metric quality will be addressed using
statistics, learning theory, and optimization on 
real-world workloads, and evaluated using both simulations and experiments to assess how well they describe real workloads.

\item[2.]
\emph{What if workloads were automatically detectable and classified into
\textit{model workload} categories?}
The PI proposes to develop a suite of algorithms to identify workloads, defined as correlated I/O activity, in traces and associate the workloads found with corresponding model
workloads.  The PI will focus on isolating interleaved workloads within static and dynamic traces.  Once workloads are isolated, they will be matched to the appropriate model category through defining a weighted similarity function.  The PI will design an online system to calculate metrics on workloads to produce hints for tuning elastic storage systems.
%proposes to primarily identify workloads using
%filesystem snapshots, which can be collected during periods of system downtime.
The study involves developing low overhead trace collection for model workload
features, separation of interleaved workloads, and building a web-based tool to provide workload analysis to the commmunity 
%to both gather training data and provide a workload visualization and annotation tool to the community.
%and system-aware model fitting
%for the workloads that are isolated.  The efficacy of the classification will be tested by interleaving %known distinct workloads and verifying that the workloads are correctly isolated and labeled with the model workload that is most similar.  

% FIXME: Include meta-grouping here for deriving workloads?

%This will be evaluated based on
%performance improvements based on the model workload system parameters being
%applied to the system.


\item[3.]
\emph{What if storage system tuning was automatic and portable?} The PI aims to develop a
parametric model of the relationship between storage workloads and the
infrastructures developed to serve them.  %Given that model workloads are defined by a set of metrics, 
The PI will architect a system that maps metrics to tuning hints while accounting for inter and intra-workload metric interactions.  

The PI aims to
demonstrate, both analytically and through experimental validation, that tunings for similar workloads are transferable between storage
systems and that automated tuning improves storage system provisioning in terms of performance and cost.
%are
%more power and cost efficient than systems designed around traditional
%constraints.
  
%\emph{What if data layout was workload-reactive? } 
%The PI aims to
%demonstrate, both analytically and through experimental validation, that storage
%systems designed to meet requirements automatically generated from workloads are
%more power and cost efficient than systems designed around traditional
%constraints. This improvement will be both direct, in the form of procedurally
%generated system requirements for a predicted level of performance, and indirect
%, in the form of better simulated workloads for system testing.  Additionally,
%strict workload characterizations will drive administrative tuning and aide hardware
%procurement and workload placement decisions.

%\emph{What if caches were cost conscious?} The PI aims to generate a systematic understanding of how external costs should be incorporated into cache systems, including
%diverse data-processing costs and performance feedback from external systems affected by cache replacement decisions.
%The study involves devising performance models as these parameters vary, harnessing properties
%of cache algorithms to reduce overhead, and are evaluated based on simulations and analysis of their predictive accuracy.

%%\textbf{(iii)}
%\emph{What if storage system tuning was infrastructure aware?} The PI
%\emph{What if caches were memory aware?} The PI seeks to create a practical theory and programming framework of how data can be represented with fewer memory resources, %in terms of computation and network instead of only memory resources, 
%%can be exploited in the context of caching, devising methods for 
%devising methods for representing cached data in formats that rather use computation or network 
%resources, and then automatically regenerating the data when needed.
%%The challenge of diverse content is resolved through regenerative program snippets for common patterns. %, such as scripts
%%for rederiving data from other cached items.
%\end{itemize}
%The program also implements an \textbf{educational plan} that exposes undergraduates, high school and K-12 students
% to research in computer systems.
%Interactive course materials on caching will be developed
%and piloted for undergraduates at Emory University, and released on a website for use at other schools.
%Cache topics will be integrated in a 
%new teaching An interactive curriculum will be developed 
%The online curriculum features an interactive testbed for analyzing cache systems \textbf{
%developed by the research program}.

%
%\item[1.]
%\textbf{(i)} 
%\emph{What if caches could learn?} The PI proposes to develop and demonstrate a theory of cache learning, a novel concept of exploiting content features of cache items to predict which cached data should be more quickly evicted for optimal cache performance.
%\emph{What if storage system tuning was portable?} The PI proposes to define a
%parameterization
%\emph{What if the cost of data layout was workload-quantifiable?} The PI
%proposes to develop a suite of unsupervised methods to learn features from
%static and dynamic workload
%traces.  These features will be ranked based on their relevance to the
%\systemfit: a measure of how closely the needs of the workload match the
%abilities of a target storage system.   Signatures of features will redefine current workload
%classifications.
%% Below lines are Ymir's, but they're totally applicable to me, and he'd had
%% them commented out.  ASK!!!

\end{myitemize}

\subsubsection*{Intellectual Merit: }

%The research outcomes of the proposal expand computer systems research in several ways.
%There is currently no quantitative method to determine what features define a
%storage workload. 



A method to select
predictive, high-information metrics to describe modern workloads coupled with a way of isolating ``functional workloads'' within a system will open a new area of inquiry for storage system tuning and design.  Questions such as ``is it worth 1 week of engineer time to reduce this latency?'' or ``should we add an SSD to a customer system'' will be easier to answer with a parametric worklaod model to base simulations from.% from existing traces will open a new area of
%inquiry to better design storage systems for a modern workload.
%Automatically determining workload features is critical
%because of the recent shift from dedicated storage systems to shared systems,
%where large amounts of data are placed together and accessed by a dynamic set of
%users and applications.
%
%Obtaining test data with particular qualities is often the most difficult part of
%designing new storage systems or algorithms.
Quantitative methods for determining workload features from static and dynamic
traces will also provide a template for designing test workloads, where the
over-fitting of the workload to the system is precisely defined and can be
adjusted based on the design requirements.

%in storage systems research
%to determine what workload features specific
%architectures are designed for.
% (i) shining the spotlight on a component that has tradition all
%a method of determining features of a workload will open up new directions in
%workload analysis and shaping
%\systemfit, and correspondingly improve .
%that differ from 
%modern distributed cache applications differ fundamentally from assumptions that have accumulated over four decades of cache research,
%the cache abstraction has been refined over the past four decades for low-level system environments 
%including %significantly 
%changes in workloads, scale, performance requirements and memory constraints.
%%

%sets of automatically learned features will allow for a new classification of
%workloads, which will lead to better communication between academia and industry
%as to the requirements and constraints on storage system design.  

% FIXME something about how awesome it is that we have a new metric of system
% fit so we can share workloads better.

\subsubsection*{Broader Impacts: }
As part of this project, the PI will build a 
web-based trace repository that will include a visualization and analysis tool. This tool will calculate metrics and
offer configuration suggestions to users while increasing the number
of traces available to the academic community.  The PI will work with SNIA to ensure that traces are widely available and a consistent trace format is maintained.

The PI will also pioneer a curriculum to attract students from the mathematics and data science communities to systems research.  The curriculum will include a workshop at the Atlanta Science Festival to introduce local high school students to quantitative systems research.  This will both bring a needed mathematical rigor to experimental systems research while improving representation for women and minorities.%, who tend to choose more mathematical CS subfields, in systems.  

%* Start a program to train more mathematically inclined people in data analysis
%for distibuted systems.  Data science is super hot, the overlap between people
%who know data science and systems is very low, and this would be a good way to
%get more women into systems since statistically women have preferred more
%quantitative areas of CS like ML and Algorithms and pure math to systems and
%infrastructure, to the detriment of the field.%

% FIXME OLD BELOW THIS
%An established set of metrics and algorithms to tune them will help translate advances in storage
%%design across systems by making storage optimization a modular enterprise.
%Additionally, a common workload characterization will improve communication of
%new discoveries across storage systems research, as researchers will be able to
%discuss the workloads in their studies in aggregate without having to
%reveal any proprietary access data.

%Improving \systemfit will allow the development of storage solutions that are
%better provisioned for the particular likely workloads.
%This will lead to systems with lower power usage and correspondingly lower cost,
%making it possible to store more data with higher levels of reliability in
%resource constrained environments.

%Common language to discuss systems workloads will be key to future efforts
%across the field in sharing datasets and determining how to fairly compare
%systems with different data.
%
%“System Fit” would allow a storage designer to easily determine whether the
%system matched the workload in a meaningful way.
%
%Future systems designs would have a common language to describe the load they
%are designing for, and would be better optimized to the real needs of that load.  
%
%Customers, both in government and industry, would be better able to communicate
%their needs to storage system designers.
%
%Cache operators stand to save both time and resources by shifting away from one-size-fits-all cache replacement and manual incorporation of priorities
%based on performance.
% FIXME
% Our thing lets us throw less data away (design bigger, better storage)
% Lets us better share the resources that we have, since we will be better able
% to articulate our requests in terms of expected resource needs and
% requirements
% lets us save kittens.  Kittens rocks.
%The research outcomes of the project 

%The project will shift the cache paradigm towards endowing system caches with greater intelligence and interaction with
%their environments,
%opening doors for further scientific discovery into making systems components smarter.
%\Smartcache{} addresses mounting data center performance problems by enabling operators to integrate performance-aware caches with modern systems ranging from
%large-scale web services to databases, reducing both programmatic and 
%operational complexity. 
%Because improved cache success rates alleviate back-end system load and reduce response time for end users, \smartcache{}'s
%efficiency translates to cost savings for businesses.
%

%FIXME 
% We have a lot of experience in workload analysis, I/O simulation, and a lot of
% industry contacts to learn what the needs are.  Additionally, at Emory we
% have lots of interplay with the needs of the life sciences, a workload that
% hadn't traditionally been served by the storage system community.

%%This project is supported by the extensive experience of the PI in
%%spatio-temporal workload analysis, including machine learning techniques to

\remove{%{{{
The project XXX

Minority high school and K-12 students will exposed to computer systems research through science fairs

The education outcomes of the project include outreach activities at events attended by high school and K-12 students 
of underrepresented groups.
 
of un
at events
well attended by sunderrepresented groups.
particularly include 


prepare a pipeline of students 
%The project is tightly integrated with educational goals and outreach

3) Systematic education on virtualization and cloud computing that harnesses the research outcomes to provide training in virtualization and cloud computing, including new education activities for graduate, undergraduate, and K-12 students, as well as a new virtual-machine-based online education system to facilitate these activities.

This project's research outcomes will enable virtualized systems to support performance guarantees for modern applications with dynamic and complex behaviors. As a result, a broader range of applications with different QoS requirements will benefit from cloud computing, and cloud services will be able to offer their users more economical QoS-based charging models instead of the currently used resource-capacity-based models. 

This project's education outcomes will enable systematic education on virtualization and cloud computing from K-12 to undergraduate and graduate classrooms and preparing a pipeline of students equipped with the necessary knowledge and skills in these emerging technologies and prepared to contribute in the coming cloud computing era.

|

%Help expand the scale of cybersecurity education.

%open-source

%Ethics


%\emph{Maximum of 1 page}

%\end{center}
%\end{justify}
% This should be a brief statement of the problem you plan to address.
% It should look something like an abstract. 

%The project summary should be a description of the proposed activity suitable
%for publication, no more than one page in length. It should not be
%an abstract of the proposal, but rather a self-contained description of
%the activity that would result if the proposal were funded. The summary
%should be written in the third person and include a statement of objectives
%and methods to be employed. It should be informative to other persons
%working in the same or related fields and understandable to a scientifically
%or technically literate lay reader. \\
%
%The summary must clearly address in separate statements (within the one-page summary):
%the intellectual merit of the proposed activity; and the broader impacts
%resulting from the proposed activity. Proposals that do not separately
%address both criteria within the one-page Project Summary will be returned without
%review. \\

%\required{Intellectual Merit}
%\subsection*{Intellectual Merit}
% This is why your project is interesting and will help further
% knowledge in the field of mathematics. 


%How important is the proposed activity to advancing
%knowledge and understanding within its own field or across different fields?
%How well qualified is the proposer (individual or team) to conduct the project?
%(If appropriate, the reviewer will comment on the quality of prior work.)
%To what extent does the proposed activity suggest and explore creative, original,
%or potentially transformative concepts? 
% How well conceived and organized is the
%proposed activity? Is there sufficient access to resources?  \\
%	Will the project produce exemplary material, processes, or models that enhance student learning? 
%	Will evaluation and research projects yield important findings related to student learning? 
%	Does the project build on existing knowledge about STEM education? 
%	Are appropriate expected measurable outcomes explicitly stated and are they integrated into an evaluation plan? 
%	Is the evaluation effort likely to produce useful information?

%Cybersecurity education XXX



%\required{Broader Impacts}
\subsection*{Broader Impacts}
The project aims to expand the scale of cybersecurity education in
high-schools, colleges and universities, both by providing an independent
learning and training resource for students and by augmenting curricula with a hands-on platform for
teaching practical cybersecurity. 
%
The online platform is accessible to diverse groups of people, providing
opportunity for women and other underrepresented minorities to participate and
develop their skills in a safe environment without the stigma often associated
with classroom settings and physical competitions. % XXX must verify and cite
%
%In addition to regular tournaments that will be hosted on \core during the
%project, the hosted and open-source \core platform facilitates independent
%deployment of Cyberdefense and Capture-The-Flag (CTF) tournaments that are
%increasingly being used by educators to enrich understanding and competency of
%Cybersecurity.
%
Finally, \core's interplay between educational lectures and hands-on exercises,
which effectively demonstrate the learning outcomes of the material, 
creates an environment for advancing cybersecurity pedagogy.
}


% There are 4 kinds of broader impacts.
% 1. advance discovery and understanding while promoting teaching,
% training and learning
% 2. broaden the participation of underrepresented groups
% 3. disseminated broadly to enhance scientific and technological
% understanding
% 4. benefits of the proposed activity to society
%  Integrate research activities into the teaching of science, math and engineering at
%all educational levels (e.g., K-12, undergraduate science majors, non-science
%majors, and graduate students).
%• 	Include students (e.g., K-12, undergraduate science majors, non-science majors,
%and /or graduate students) as participants in the proposed activities as appropriate.
%• 	Participate in the recruitment, training, and/or professional development of K-12
%science and math teachers.
%• 	Develop research-based educational materials or contribute to databases useful in
%teaching (e.g., K-16 digital library).
%• 	Partner with researchers and educators to develop effective means of
%incorporating research into learning and education.
%	Establish special mentoring programs for high school students, undergraduates,
%graduate students, and technicians conducting research.
%• 	Involve graduate and post-doctoral researchers in undergraduate teaching activities.
%•	Develop, adopt, adapt or disseminate effective models and pedagogic approaches
%to science, mathematics and engineering teaching.
%
%
% Participate in developing new approaches (e.g., use of information technology and
%connectivity) to engage underserved individuals, groups, and communities in science and engineering.
% Make campus visits and presentations at institutions that serve underrepresented
%groups.
%	Establish research and education collaborations with faculty and students at
%community colleges, colleges for women, undergraduate institutions, and EPSCoR institutions.
%
%
%
% 	Identify and establish collaborations between disciplines and institutions, among
%the U.S. academic institutions, industry and government and with international
%partners.
%	Stimulate and support the development and dissemination of next-generation
%instrumentation, multi-user facilities, and other shared research and education
%platforms.
%	Maintain, operate and modernize shared research and education infrastructure,
%including facilities and science and technology centers and engineering research
%centers.
%	Upgrade the computation and computing infrastructure, including advanced
%computing resources and new types of information tools (e.g., large databases,
%networks and associated systems, and digital libraries).
%	Develop activities that ensure that multi-user facilities are sites of research and
%mentoring for large numbers of science and engineering students.
%Involve the public or industry, where possible, in research and education activities.
%Give science and engineering presentations to the broader community (e.g., at
%	museums and libraries, on radio shows, and in other such venues).
%Make data available in a timely manner by means of databases, digital libraries, or
%	other venues such as CD-ROMs.
% Integrate research with education activities in order to communicate in a broader
% 	context.
% 
%
% 	Demonstrate the linkage between discovery and societal benefit by providing
%specific examples and explanations regarding the potential application of research
%and education results.
%	Partner with academic scientists, staff at federal agencies and with the private
%sector on both technological and scientific projects to integrate research into
%broader programs and activities of national interest.
%	Analyze, interpret, and synthesize research and education results in formats
%understandable and useful for non-scientists.
%
%
%  have a broad impact on STEM education in an area of recognized need or opportunity? 
%	have the potential to contribute to transformative change in undergraduate STEM education?
%
% 	Broadening opportunities and enabling the participation of all citizens -- women and men, underrepresented minorities, and persons with disabilities -- is essential to the health and vitality of science and engineering. %}}}


